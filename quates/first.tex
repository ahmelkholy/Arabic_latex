\documentclass{article}
\usepackage[paperwidth=4in, paperheight=2in, margin=0pt]{geometry}
\usepackage{polyglossia}
\usepackage{hyperref}
\usepackage{geometry}
\usepackage{xcolor}

\geometry{margin=0pt}
\pagecolor{black}
\color{white}

\setmainlanguage{english}
\setotherlanguage{arabic}
\newfontfamily\arabicfont[Script=Arabic]{Amiri}

\begin{document}
\begin{Arabic}

 اليوْم تُوفِّي الشَّارخ اَلذِي أَعطَى اَللغَة العربيَّة فُرصَة دُخُول عَالَم الحاسوب وأيْضًا جُزْء مِن تَارِيخ المشاريع الفرْديَّة وغياب الاهْتمام اَلشعْبِي بِهَا بِمعْنى أَنهَا لَم تَكُن مَنظُومة ساندتْهَا الشُّعوب رحل أحد صَاحِب المشاريع اَلمهِمة فِي حَيَاة الشُّعوب النَّاطقة والْمحبَّة لِلْعربيَّة وَفِي اِنتِظار آخر لِمشْروع آخر يَنتَهِي عِنْد رحيل القادم إِذَا قَدَّم
 
 فِي يَوْم رَحيلِه نَعرُج إِلى \href{https://archive.alsharekh.org/}{الأرْشيف} اَلذِي بَنَاه لِيتيح النَّوادر مِن المصنَّفات العربيَّة وَالذِي يَحمِل اِسْمه فِيه تُصَاب مِن النَّسْتولْجَا مِن شَكْل المجلَّات وأيْضًا تَتَعجَّب مِن قُدرَة المجْهودات الفرْديَّة على صُنْع إِرْث 

\end{Arabic}
\end{document}
